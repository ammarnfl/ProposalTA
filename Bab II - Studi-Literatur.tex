% ==========================================
% BAB II STUDI LITERATUR
% ==========================================
\chapter{STUDI LITERATUR}
\label{chap:studi-literatur}
\section{Program Makan Bergizi (MBG)}

Program Makan Bergizi Gratis (MBG) merupakan inisiatif pemerintah Indonesia untuk meningkatkan kesehatan dan kesejahteraan siswa melalui penyediaan makanan bergizi secara rutin di sekolah. Program ini mencakup seluruh tingkat pendidikan dari PAUD hingga SMA/SMK, dengan target 82{,}9 juta penerima hingga tahun 2029.

Tujuan utama program ini adalah untuk mengatasi permasalahan gizi buruk, meningkatkan konsentrasi belajar, serta mendukung perkembangan fisik dan mental anak usia sekolah. Pada tingkat SMA/SMK, program ini memiliki karakteristik khusus karena siswa sudah memiliki kemampuan untuk memberikan \textit{feedback} yang lebih terstruktur dan menggunakan teknologi digital secara lebih mandiri.

Dalam praktiknya, beberapa permasalahan masih sering muncul, seperti distribusi makanan yang tidak merata, kualitas makanan yang tidak konsisten, keterbatasan dalam proses \textit{monitoring} dan evaluasi di lapangan, serta minimnya mekanisme untuk mengumpulkan \textit{feedback} langsung dari pengguna (siswa, guru, dan orang tua). Kondisi ini menunjukkan perlunya sistem yang dapat memfasilitasi komunikasi dua arah antara penyedia layanan dan pengguna akhir (\textbf{Kementerian Pendidikan dan Kebudayaan, 2023}).

Untuk menjamin keberlanjutan dan efektivitas program, dibutuhkan sistem informasi yang mampu memantau distribusi makanan, mendeteksi keluhan, serta mengevaluasi kepuasan pengguna secara \textit{real-time}. Dengan demikian, sistem berbasis teknologi informasi dapat membantu pihak sekolah dan pemerintah dalam mengambil keputusan berbasis data (\textit{data-driven decision making}) serta meningkatkan responsivitas terhadap kebutuhan pengguna.

\section{Sistem Informasi \textit{Monitoring}}

Sistem informasi \textit{monitoring} merupakan aplikasi berbasis komputer yang dirancang untuk mengumpulkan, mengolah, dan menampilkan data kegiatan tertentu secara \textit{real-time} sehingga pihak pengambil keputusan dapat melakukan evaluasi dengan cepat dan tepat.

Dalam konteks Program MBG, sistem informasi \textit{monitoring} dapat digunakan untuk:
\begin{enumerate}
    \item Mencatat distribusi makanan dan waktu penyajian.
    \item Mengumpulkan data presensi konsumsi siswa.
    \item Menyimpan dan menampilkan laporan keluhan dari siswa atau guru.
    \item Menyediakan visualisasi dan analisis tren kualitas makanan.
\end{enumerate}

Penelitian sebelumnya menunjukkan bahwa implementasi sistem \textit{monitoring} berbasis \textit{web} dan \textit{mobile} dapat meningkatkan transparansi, efisiensi, serta akurasi data dalam kegiatan operasional sekolah (\textbf{Suryani \& Prasetyo, 2021}). Sistem \textit{monitoring} yang baik juga memfasilitasi komunikasi antara berbagai \textit{stakeholder} (guru, siswa, orang tua, dan pihak dapur) dalam lingkup program yang diawasi.

Khusus untuk tingkat SMA/SMK, siswa umumnya memiliki kemampuan teknologi yang memadai untuk berinteraksi dengan sistem digital, memberikan \textit{feedback} yang lebih detail, dan menggunakan aplikasi \textit{mobile} secara mandiri. Hal ini membuat implementasi sistem \textit{monitoring} di tingkat ini lebih efektif dibandingkan tingkat pendidikan yang lebih rendah.

\section{Analisis Sentimen dan \textit{Feedback Management}}

Analisis sentimen (\textit{sentiment analysis}) adalah teknik pemrosesan bahasa alami (\textit{Natural Language Processing}) yang digunakan untuk mengidentifikasi dan mengklasifikasikan emosi atau opini yang terkandung dalam teks. Dalam konteks manajemen \textit{feedback}, analisis sentimen dapat membantu organisasi memahami tingkat kepuasan pengguna secara otomatis dan sistematis.

Dalam sistem MBG \textit{Monitoring}, kecerdasan buatan (\textit{Artificial Intelligence}) dapat diterapkan pada beberapa aspek, antara lain:
\begin{enumerate}
    \item Rekomendasi menu sehat, menggunakan algoritma \textit{machine learning} untuk menyesuaikan menu makanan berdasarkan kebutuhan gizi siswa dan preferensi lokal.
    \item \textit{Anomaly detection}, yaitu mendeteksi pola tidak normal seperti meningkatnya keluhan setelah menu tertentu atau distribusi makanan yang terlambat.
    \item \textit{Sentiment analysis}, yaitu menganalisis \textit{feedback} dari siswa dan guru terhadap makanan yang disajikan untuk mengetahui tingkat kepuasan.
\end{enumerate}

Studi oleh \textbf{Zhang et al.\ (2022)} menunjukkan bahwa kombinasi antara sistem informasi dan algoritma pembelajaran mesin dapat meningkatkan efisiensi program kesehatan masyarakat hingga 30\%.

\section{Penelitian Terdahulu yang Relevan}

Beberapa penelitian yang relevan dengan topik ini antara lain:
\begin{enumerate}
    \item \textbf{Putra dan Sari (2022)} mengembangkan sistem \textit{monitoring} gizi siswa berbasis \textit{mobile} yang memungkinkan guru melaporkan data konsumsi makanan secara \textit{real-time}.
    \item \textbf{Rahmawati et al.\ (2021)} mengusulkan sistem deteksi dini kasus gizi buruk menggunakan algoritma \textit{Decision Tree} dengan tingkat akurasi mencapai 89\%.
    \item \textbf{Santoso (2020)} merancang sistem evaluasi kualitas makanan di sekolah berbasis \textit{Internet of Things} (IoT) yang mengirimkan data suhu dan kesegaran makanan ke \textit{server} pusat.
\end{enumerate}

Meskipun penelitian-penelitian tersebut menunjukkan kemajuan signifikan, belum ada sistem yang secara komprehensif mengintegrasikan \textit{monitoring} distribusi, presensi konsumsi, keluhan pengguna, serta analisis berbasis AI secara bersamaan. Hal ini menjadi \textit{research gap} dan dasar pengembangan sistem pemantauan MBG pada penelitian ini.

\section{Kerangka Konseptual}

Kerangka konseptual dalam penelitian ini menggambarkan keterkaitan antara komponen sistem MBG \textit{Monitoring} dengan elemen AI yang diterapkan. Secara umum, sistem akan terdiri dari:

\begin{enumerate}
    \item \textit{Input}: data distribusi makanan, data presensi siswa, data keluhan, dan data menu harian;
    \item \textit{Proses}: analisis pola konsumsi menggunakan algoritma \textit{machine learning} (rekomendasi menu dan deteksi anomali);
    \item \textit{Output}: \textit{dashboard} \textit{monitoring}, notifikasi peringatan, dan rekomendasi perbaikan menu.
\end{enumerate}

Dengan demikian, sistem ini diharapkan dapat mendukung pelaksanaan Program MBG secara efisien, transparan, dan adaptif terhadap kondisi nyata di lapangan.

\section{Kerangka Teori}

Kerangka teori menjelaskan konsep dan teori yang menjadi landasan pengembangan sistem pemantauan MBG dengan AI. Bagian ini mencakup teori-teori tentang sistem informasi, kecerdasan buatan, \textit{machine learning}, deteksi anomali, serta rekomendasi berbasis data.

\subsection{Teori Sistem Informasi}

Teori sistem informasi menjelaskan bagaimana sistem yang terkomputerisasi dapat mengintegrasikan berbagai data, proses, dan pengguna untuk mencapai tujuan organisasi. Sistem informasi yang baik memiliki karakteristik akurat, relevan, mudah diakses (\textit{accessible}), \textit{real-time}, dan aman (\textit{secure}).

\subsection{Teori Kepuasan Pengguna (\textit{User Satisfaction Theory})}

Kepuasan pengguna merupakan salah satu ukuran utama keberhasilan sebuah sistem informasi. \textbf{DeLone dan McLean (2003)} mengembangkan model kesuksesan sistem informasi yang mencakup kualitas sistem, kualitas informasi, penggunaan sistem, kepuasan pengguna, dan manfaat bersih. Dalam konteks sistem MBG, kepuasan pengguna dapat diukur melalui:
\begin{enumerate}
    \item Kemudahan penggunaan sistem,
    \item Ketersediaan dan akurasi informasi,
    \item Responsivitas terhadap \textit{feedback} pengguna.
\end{enumerate}

\section{Kesimpulan Studi Literatur}

Berdasarkan tinjauan pustaka di atas, dapat disimpulkan bahwa:
\begin{enumerate}
    \item Program MBG membutuhkan mekanisme \textit{monitoring} yang lebih efektif dan transparan.
    \item Sistem informasi \textit{monitoring} berbasis digital dapat meningkatkan efisiensi operasional dan transparansi.
    \item Analisis sentimen menggunakan \textit{machine learning} merupakan teknik yang \textit{feasible} dan memberikan nilai tambah dalam memahami kepuasan pengguna.
    \item Belum ada sistem yang secara komprehensif mengintegrasikan \textit{monitoring} MBG dengan \textit{sentiment analysis} secara \textit{real-time}.
    \item Pengembangan sistem pemantauan MBG dengan \textit{sentiment analysis} merupakan penelitian yang relevan dan inovatif.
\end{enumerate}