% ==========================================
% BAB IV DESAIN KONSEP SOLUSI
% ==========================================
\chapter{DESAIN KONSEP SOLUSI}
\label{chap:desain-konsep-solusi}

\section{Desain Konsep Solusi}

Setelah mengidentifikasi kebutuhan sistem pada bab sebelumnya, tahap berikutnya adalah merancang solusi yang akan dikembangkan. Bagian ini menyajikan desain konsep solusi berupa model konseptual, perbandingan dengan sistem saat ini, arsitektur sistem, serta desain antarmuka yang diusulkan.

Proses perancangan sistem MBG \textit{Monitoring} mengikuti metodologi \textit{structured design} yang terdiri dari beberapa fase iteratif. Setiap fase menghasilkan rangkaian desain yang akan menjadi dasar untuk fase berikutnya. Bagian ini menjelaskan tahapan-tahapan tersebut secara sistematis.

\subsection{Model Konseptual Sistem Saat Ini}

Sistem Program Makan Bergizi Gratis (MBG) di SMA saat ini mengikuti alur operasional yang masih bersifat manual dan belum terintegrasi. Model konseptual sistem tersebut ditunjukkan pada Gambar~\ref{fig:bab4-model-saat-ini}.

Alur operasional sistem saat ini memiliki beberapa kendala utama sebagai berikut.
\begin{enumerate}[a.]
  \item Data tidak terintegrasi.
  \item \textit{Feedback} tidak terdokumentasi.
  \item Analisis dilakukan secara manual dan memakan waktu.
  \item Sulit melacak tren permasalahan secara konsisten.
\end{enumerate}

\begin{figure}[h]
  \centering
  \captionsetup{justification=centering}
  \includegraphics[width=0.5\textwidth]{image/bab4-model-saat-ini}
  \caption{Model konseptual sistem MBG saat ini}
  \label{fig:bab4-model-saat-ini}
\end{figure}

\FloatBarrier

\subsection{Model Konseptual Sistem Usulan}

Sistem MBG \textit{Monitoring} yang diusulkan mengintegrasikan teknologi informasi untuk mengotomatisasi proses pencatatan, \textit{feedback management}, dan analisis sentimen. Model konseptual sistem usulan ditunjukkan pada Gambar~\ref{fig:bab4-model-usulan}.

Alur operasional sistem usulan ini memiliki beberapa kelebihan sebagai berikut.
\begin{enumerate}[a.]
  \item Data terintegrasi dan dapat diakses secara \textit{real-time}.
  \item \textit{Feedback} dapat dianalisis secara otomatis.
  \item Tren dan pola keluhan lebih mudah diidentifikasi.
  \item Pengambilan keputusan dapat dilakukan secara lebih objektif dan berbasis data.
\end{enumerate}

\begin{figure}[h]
  \centering
  \captionsetup{justification=centering}
  \includegraphics[width=0.4\textwidth]{image/bab4-model-usulan}
  \caption{Model konseptual sistem MBG usulan}
  \label{fig:bab4-model-usulan}
\end{figure}

\FloatBarrier

\section{Perbandingan Sistem Saat Ini dan Sistem Usulan}

\subsection{Perbandingan Fitur dan Fungsionalitas}

Setelah menganalisis model konseptual antara sistem yang sudah ada dengan sistem usulan, dilakukan perbandingan fitur dan fungsionalitas utama. Ringkasan perbandingan antara sistem saat ini (manual) dan sistem usulan yang terintegrasi disajikan pada Tabel~\ref{tab:bab4-perbandingan-sistem}.

\begin{longtable}{| p{0.20\textwidth} | p{0.36\textwidth} | p{0.36\textwidth} |}
\caption{Perbandingan sistem MBG saat ini dan sistem usulan}
\label{tab:bab4-perbandingan-sistem} \\
\hline
Aspek & Sistem saat ini & Sistem usulan \\
\hline
\endfirsthead

\multicolumn{3}{l}{\textit{Tabel \ref{tab:bab4-perbandingan-sistem} (lanjutan)}}\\
\hline
Aspek & Sistem saat ini & Sistem usulan \\
\hline
\endhead

\hline
\multicolumn{3}{r}{\textit{Bersambung ke halaman berikutnya}} \\
\endfoot

\hline
\endlastfoot

Pencatatan distribusi &
Tidak ada pencatatan distribusi yang terdokumentasi dengan baik. &
Pencatatan distribusi dilakukan secara \textit{digital} dan terpusat secara \textit{real-time}. \\
\hline
Presensi konsumsi &
Presensi hanya sebatas daftar hadir manual dan sulit dilacak kembali. &
Presensi terintegrasi dengan sistem sehingga partisipasi konsumsi mudah ditelusuri. \\
\hline
Pengumpulan \textit{feedback} &
\textit{Feedback} disampaikan secara lisan dan tidak terdokumentasi. &
\textit{Feedback} dikumpulkan melalui aplikasi dan tersimpan dalam basis data. \\
\hline
Analisis \textit{feedback} &
Analisis dilakukan secara manual sehingga memakan waktu dan berpotensi bias. &
Analisis dilakukan secara otomatis dengan bantuan AI sehingga lebih objektif dan konsisten. \\
\hline
Kategorisasi keluhan &
Tidak ada kategorisasi; laporan keluhan tercampur dan sulit dikelompokkan. &
Keluhan dikategorikan secara otomatis oleh sistem berdasarkan isi \textit{feedback}. \\
\hline
\textit{Monitoring} program &
Dilakukan secara periodik dan informal, bergantung pada pertemuan manual. &
Dilakukan secara \textit{real-time} melalui \textit{dashboard} yang selalu tersedia. \\
\hline
Transparansi program &
Transparansi rendah; siswa tidak mengetahui progres program secara \textit{real-time}. &
Transparansi tinggi; siswa dapat melihat jadwal, menu, dan status \textit{feedback} secara \textit{real-time}. \\
\hline
Pengambilan keputusan &
Berbasis intuisi dan pengalaman personal. &
Berbasis data dan \textit{insight} yang dihasilkan AI. \\
\hline

\end{longtable}


\FloatBarrier

\subsection{Perbandingan Alur Proses \textit{Feedback}}

Dalam sistem usulan, proses pelaporan dan penanganan \textit{feedback} dapat dilakukan secara langsung oleh siswa dan guru melalui sistem, kemudian diproses secara otomatis oleh komponen AI untuk pengkategorisasian dan analisis sentimen. Hasil analisis dapat dipantau secara \textit{real-time} oleh operator dapur.

Perbandingan alur pelaporan program antara sistem saat ini dan sistem usulan ditunjukkan pada Gambar~\ref{fig:bab4-alur-feedback-saat-ini} dan Gambar~\ref{fig:bab4-alur-feedback-usulan}.

\begin{figure}[h]
  \centering
  \captionsetup{justification=centering}
  \includegraphics[width=0.7\textwidth]{image/bab4-alur-feedback-saat-ini}
  \caption{Alur pelaporan program pada sistem saat ini}
  \label{fig:bab4-alur-feedback-saat-ini}
\end{figure}

\FloatBarrier

\begin{figure}[h]
  \centering
  \captionsetup{justification=centering}
  \includegraphics[width=0.7\textwidth]{image/bab4-alur-feedback-usulan}
  \caption{Alur pelaporan program pada sistem usulan}
  \label{fig:bab4-alur-feedback-usulan}
\end{figure}

\FloatBarrier

\section{Use Case Sistem Usulan}

\subsection{Analisis \textit{Use Case}}

Analisis \textit{use case} dilakukan untuk memahami interaksi antara pengguna dan sistem, serta mengidentifikasi fungsi-fungsi utama yang harus disediakan. Diagram \textit{use case} mengilustrasikan aktor yang terlibat serta fungsionalitas yang dapat mereka akses. Perancangan hubungan antara fungsionalitas ini akan memudahkan proses perancangan antarmuka, alur proses, serta prioritas implementasi.

Berdasarkan analisis kebutuhan fungsional pada Bab~\ref{chap:analisis-masalah}, dapat diidentifikasi aktor-aktor berikut yang akan berinteraksi dengan sistem.
\begin{enumerate}
  \item Siswa, sebagai aktor utama penerima manfaat program yang mengisi presensi konsumsi, memberikan \textit{feedback} dan \textit{rating} makanan, serta mengakses informasi menu dan tindak lanjut keluhan.
  \item Guru pembina, sebagai aktor yang memantau presensi konsumsi siswa, mengirim observasi atau keluhan tambahan, serta memanfaatkan \textit{dashboard} dan laporan untuk evaluasi program di tingkat kelas atau sekolah.
  \item Operator dapur, sebagai aktor yang mengelola data menu dan distribusi makanan, memonitor \textit{feedback} dan \textit{rating} dari siswa dan guru, serta menggunakan ringkasan analitik untuk memperbaiki kualitas layanan.
\end{enumerate}

Berdasarkan ketiga aktor tersebut dan tujuh belas kebutuhan fungsional yang telah ditetapkan, disusun daftar \textit{use case} sistem seperti ditunjukkan pada Tabel~\ref{tab:bab4-usecase-sistem}.


\begin{longtable}{| p{0.08\textwidth} | p{0.26\textwidth} | p{0.58\textwidth} |}
\caption{\textit{Use case} sistem MBG \textit{Monitoring}}
\label{tab:bab4-usecase-sistem} \\
\hline
Kode & \textit{Use case} & Deskripsi \\
\hline
\endfirsthead

\multicolumn{3}{l}{\textit{Tabel \ref{tab:bab4-usecase-sistem} (lanjutan)}}\\
\hline
Kode & \textit{Use case} & Deskripsi \\
\hline
\endhead

\hline
\multicolumn{3}{r}{\textit{Bersambung ke halaman berikutnya}} \\
\endfoot

\hline
\endlastfoot

UC.1 & Login/Logout &
Semua aktor dapat melakukan \textit{login}/\textit{logout} ke dalam sistem, memverifikasi kredensial, dan memperoleh akses sesuai peran. \\
\hline
UC.2 & Mengelola menu harian &
Operator dapur menginput dan memperbarui menu harian beserta informasi porsi makanan yang akan disajikan kepada siswa. \\
\hline
UC.3 & Mencatat dan memperbarui distribusi makanan &
Operator dapur mencatat jadwal dan status distribusi (jumlah porsi keluar, porsi tersisa, dan status distribusi) yang digunakan sebagai sumber data untuk \textit{dashboard} distribusi. \\
\hline
UC.4 & Melihat informasi menu dan jadwal distribusi &
Siswa dan guru melihat menu makanan serta jadwal distribusi pada hari ini dan hari tertentu sehingga memperoleh informasi program MBG secara transparan. \\
\hline
UC.5 & Mengisi presensi konsumsi &
Siswa mencatat secara mandiri makanan yang dikonsumsi pada hari tersebut. \\
\hline
UC.6 & Memverifikasi dan melihat statistik presensi &
Guru memverifikasi atau mengoreksi presensi konsumsi bila diperlukan, serta melihat statistik presensi per kelas atau per hari untuk memantau partisipasi siswa. \\
\hline
UC.7 & Memberikan \textit{feedback} dan \textit{rating} makanan &
Siswa mengisi formulir \textit{feedback} teks dan memberikan \textit{rating} numerik terhadap makanan yang diterima (skala 1--5). \\
\hline
UC.8 & Melihat riwayat \textit{feedback} dan tindak lanjut &
Siswa melihat daftar \textit{feedback} dan \textit{rating} yang pernah dikirim beserta status tindak lanjutnya (misalnya diterima, dalam peninjauan, atau sudah ditindaklanjuti). \\
\hline
UC.9 & Mengirim observasi atau keluhan guru &
Guru menyampaikan observasi tambahan atau keluhan terkait pelaksanaan program, misalnya keterlambatan distribusi atau kualitas makanan. \\
\hline
UC.10 & Mengelola \textit{feedback} dan keluhan &
Operator dapur meninjau \textit{feedback} siswa dan keluhan guru, melakukan pengelompokan, menandai status penanganan, serta mencatat langkah perbaikan yang diambil. \\
\hline
UC.11 & Melihat \textit{dashboard} distribusi &
Operator dapur dan guru melihat ringkasan visual terkait distribusi makanan hari ini, seperti jumlah porsi yang sudah dibagikan, titik hambatan, dan status distribusi per kelas atau per sesi. \\
\hline
UC.12 & Melihat \textit{dashboard} sentimen dan \textit{rating} &
Operator dapur dan guru mengakses \textit{dashboard} yang menampilkan tren \textit{rating}, ringkasan sentimen, serta kategori keluhan utama dari waktu ke waktu sebagai dasar evaluasi kualitas program. \\
\hline
UC.13 & Menghasilkan laporan evaluasi berkala &
Sistem menyusun dan menyediakan laporan berkala (misalnya mingguan atau bulanan) terkait partisipasi konsumsi, \textit{rating} makanan, dan pola keluhan, yang dapat diunduh oleh operator dapur atau guru pembina. \\
\hline
\end{longtable}


\FloatBarrier

\subsection{\textit{Use Case} Diagram}

Susunan \textit{use case} yang dikembangkan dalam sistem pemantauan Program Makan Bergizi Gratis digambarkan melalui \textit{use case} diagram pada Gambar~\ref{fig:bab4-usecase-diagram}.

\begin{figure}[h]
  \centering
  \captionsetup{justification=centering}
  \includegraphics[width=0.7\textwidth]{image/bab4-usecase-diagram}
  \caption{\textit{Use case} diagram sistem MBG \textit{Monitoring} usulan}
  \label{fig:bab4-usecase-diagram}
\end{figure}

\FloatBarrier

Diagram pada Gambar~\ref{fig:bab4-usecase-diagram} menggambarkan interaksi antara tiga aktor utama, yaitu siswa, guru pembina, dan operator dapur. Siswa berperan sebagai penerima manfaat program yang mengisi presensi konsumsi, memberikan \textit{feedback} dan \textit{rating} terhadap makanan, serta mengakses informasi menu dan riwayat tindak lanjut keluhan. Guru pembina menggunakan sistem untuk memantau presensi konsumsi, mengirim observasi tambahan, serta melihat ringkasan data melalui \textit{dashboard} dan laporan evaluasi. Sementara itu, operator dapur memanfaatkan sistem untuk mengelola menu dan distribusi makanan, meninjau serta menindaklanjuti \textit{feedback}, dan menggunakan ringkasan analitik sebagai dasar perbaikan kualitas layanan.

Selain menggambarkan hubungan antara aktor dan fungsi, \textit{use case} diagram juga menunjukkan dependensi antarfungsi melalui relasi \textit{extend}. Pemberian \textit{feedback} dan \textit{rating} makanan memperluas skenario pengisian presensi konsumsi karena umumnya dilakukan setelah siswa menyatakan telah menerima dan mengonsumsi makanan. Hubungan ini membantu dalam menyusun prioritas dan alur implementasi fitur pada sistem.

\section{Bisnis Proses}

Model proses bisnis disusun menggunakan notasi \textit{Business Process Model and Notation} (BPMN) untuk menggambarkan alur operasional harian Program Makan Bergizi Gratis (MBG) yang didukung oleh sistem \textit{monitoring} yang dirancang. BPMN digunakan untuk memperlihatkan interaksi antara tiga aktor utama, yaitu operator dapur, guru pembina, dan siswa. Model ini menggambarkan interaksi antaraktor melalui sistem dalam siklus perencanaan menu, distribusi makanan, konsumsi, pengisian presensi, pemberian \textit{feedback} dan \textit{rating}, hingga evaluasi kualitas program.

Diagram BPMN yang diusulkan dibangun dalam satu \textit{pool} ``Proses Monitoring MBG'' yang terdiri atas beberapa \textit{swimlane} untuk setiap aktor dan komponen sistem pendukung. Dengan cara ini, peran dan tanggung jawab masing-masing aktor terlihat secara jelas, sekaligus menunjukkan titik-titik integrasi yang difasilitasi oleh sistem \textit{monitoring} berbasis web. Diagram proses bisnis sistem usulan ditunjukkan pada Gambar~\ref{fig:bab4-bpmn}.

\begin{figure}[h]
  \centering
  \captionsetup{justification=centering}
  \includegraphics[width=0.7\textwidth]{image/bab4-bpmn-monitoring}
  \caption{Proses bisnis sistem MBG \textit{Monitoring} usulan}
  \label{fig:bab4-bpmn}
\end{figure}

\FloatBarrier

Gambar~\ref{fig:bab4-bpmn} menggambarkan alur proses harian pada sistem MBG \textit{Monitoring} yang melibatkan operator dapur, siswa, dan guru pembina. Operator dapur memublikasikan menu serta jadwal distribusi dan memperbarui \textit{dashboard} distribusi harian. Siswa kemudian menerima dan mengonsumsi makanan, mengisi presensi konsumsi di sistem, dan secara opsional memberikan komentar atau \textit{feedback} serta \textit{rating} terhadap makanan. Data presensi, \textit{feedback}, dan \textit{rating} yang terkumpul dimanfaatkan guru pembina untuk melihat statistik presensi konsumsi siswa, mengakses \textit{dashboard} distribusi, sentimen, dan \textit{rating}, lalu menentukan tindakan perbaikan yang perlu dilakukan terhadap pelaksanaan program MBG.