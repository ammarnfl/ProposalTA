\begin{longtable}{| p{0.20\textwidth} | p{0.36\textwidth} | p{0.36\textwidth} |}
\caption{Perbandingan sistem MBG saat ini dan sistem usulan}
\label{tab:bab4-perbandingan-sistem} \\
\hline
Aspek & Sistem saat ini & Sistem usulan \\
\hline
\endfirsthead

\multicolumn{3}{l}{\textit{Tabel \ref{tab:bab4-perbandingan-sistem} (lanjutan)}}\\
\hline
Aspek & Sistem saat ini & Sistem usulan \\
\hline
\endhead

\hline
\multicolumn{3}{r}{\textit{Bersambung ke halaman berikutnya}} \\
\endfoot

\hline
\endlastfoot

Pencatatan distribusi &
Tidak ada pencatatan distribusi yang terdokumentasi dengan baik. &
Pencatatan distribusi dilakukan secara \textit{digital} dan terpusat secara \textit{real-time}. \\
\hline
Presensi konsumsi &
Presensi hanya sebatas daftar hadir manual dan sulit dilacak kembali. &
Presensi terintegrasi dengan sistem sehingga partisipasi konsumsi mudah ditelusuri. \\
\hline
Pengumpulan \textit{feedback} &
\textit{Feedback} disampaikan secara lisan dan tidak terdokumentasi. &
\textit{Feedback} dikumpulkan melalui aplikasi dan tersimpan dalam basis data. \\
\hline
Analisis \textit{feedback} &
Analisis dilakukan secara manual sehingga memakan waktu dan berpotensi bias. &
Analisis dilakukan secara otomatis dengan bantuan AI sehingga lebih objektif dan konsisten. \\
\hline
Kategorisasi keluhan &
Tidak ada kategorisasi; laporan keluhan tercampur dan sulit dikelompokkan. &
Keluhan dikategorikan secara otomatis oleh sistem berdasarkan isi \textit{feedback}. \\
\hline
\textit{Monitoring} program &
Dilakukan secara periodik dan informal, bergantung pada pertemuan manual. &
Dilakukan secara \textit{real-time} melalui \textit{dashboard} yang selalu tersedia. \\
\hline
Transparansi program &
Transparansi rendah; siswa tidak mengetahui progres program secara \textit{real-time}. &
Transparansi tinggi; siswa dapat melihat jadwal, menu, dan status \textit{feedback} secara \textit{real-time}. \\
\hline
Pengambilan keputusan &
Berbasis intuisi dan pengalaman personal. &
Berbasis data dan \textit{insight} yang dihasilkan AI. \\
\hline

\end{longtable}
