\begin{longtable}{| p{0.07\textwidth} | p{0.22\textwidth} | p{0.43\textwidth} | p{0.18\textwidth} |}
\caption{Kebutuhan fungsional sistem}
\label{tab:kebutuhan-fungsional} \\
\hline
Kode & Kebutuhan fungsional & Deskripsi & Aktor \\
\hline
\endfirsthead

\multicolumn{4}{l}{\textit{Tabel \ref{tab:kebutuhan-fungsional} (lanjutan)}}\\
\hline
Kode & Kebutuhan fungsional & Deskripsi & Aktor \\
\hline
\endhead

\hline
\multicolumn{4}{r}{\textit{Bersambung ke halaman berikutnya}} \\
\endfoot

\hline
\endlastfoot

% ====== DATA TABEL DIMULAI DI SINI ======
F.1  & Manajemen menu harian &
Menginput menu harian, komposisi, dan jumlah porsi makanan. &
Operator dapur \\
\hline
F.2  & Pencatatan distribusi makanan &
Mencatat proses distribusi (mulai distribusi, jumlah porsi keluar, status distribusi). &
Operator dapur \\
\hline
F.3  & \textit{Dashboard} distribusi \textit{real-time} &
Menampilkan status distribusi makanan hari ini. &
Operator dapur, guru pembina \\
\hline
F.4  & Presensi konsumsi oleh siswa &
Siswa mengisi presensi konsumsi secara mandiri. &
Siswa \\
\hline
F.5  & Presensi konsumsi oleh guru &
Guru memverifikasi atau melengkapi presensi siswa jika diperlukan. &
Guru pembina \\
\hline
F.6  & Statistik presensi konsumsi &
Melihat statistik jumlah siswa yang mengonsumsi makanan. &
Guru pembina, operator dapur \\
\hline
F.7  & \textit{Form feedback} siswa &
Siswa memberikan komentar terkait kualitas makanan. &
Siswa \\
\hline
F.8  & \textit{Rating} penilaian makanan &
Siswa memberikan \textit{rating} numerik (skala 1--5). &
Siswa \\
\hline
F.9  & Riwayat \textit{feedback} dan tindak lanjut &
Siswa melihat status tindak lanjut atas \textit{feedback} mereka. &
Siswa \\
\hline
F.10 & Observasi/keluhan dari guru &
Guru melaporkan keluhan atau observasi terkait distribusi atau kualitas makanan. &
Guru pembina \\
\hline
F.11 & Manajemen \textit{feedback} oleh dapur &
Melihat, mengkategorikan, dan menindaklanjuti \textit{feedback} dan \textit{rating}. &
Operator dapur \\
\hline
F.12 & Laporan evaluasi kualitas makanan &
Menghasilkan laporan tren \textit{rating}, sentimen, dan keluhan. &
Operator dapur, guru pembina \\
\hline
F.13 & Analisis sentimen otomatis &
Sistem menganalisis sentimen \textit{feedback} siswa (positif/netral/negatif). &
Sistem AI \\
\hline
F.14 & Kategorisasi keluhan &
Sistem mengelompokkan keluhan secara otomatis. &
Sistem AI \\
\hline
F.15 & \textit{Dashboard} sentimen dan \textit{rating} &
Menampilkan grafik tren sentimen, \textit{rating} harian, dan kategori keluhan. &
Operator dapur, guru pembina \\
\hline
F.16 & Akses menu dan jadwal distribusi &
Siswa melihat menu dan jadwal distribusi yang transparan. &
Siswa \\
\hline
F.17 & \textit{Dashboard} \textit{monitoring} harian &
Guru melihat rangkuman status menu, distribusi, \textit{rating}, dan \textit{feedback} tiap harinya. &
Guru pembina \\
\hline

% ====== AKHIR DATA TABEL ======

\end{longtable}