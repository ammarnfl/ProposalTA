\begin{longtable}{| p{0.08\textwidth} | p{0.26\textwidth} | p{0.56\textwidth} |}
\caption{Kebutuhan nonfungsional sistem}
\label{tab:kebutuhan-nonfungsional} \\
\hline
Kode & Kebutuhan nonfungsional & Deskripsi \\
\hline
\endfirsthead

\multicolumn{3}{l}{\textit{Tabel \ref{tab:kebutuhan-nonfungsional} (lanjutan)}}\\
\hline
Kode & Kebutuhan nonfungsional & Deskripsi \\
\hline
\endhead

\hline
\multicolumn{3}{r}{\textit{Bersambung ke halaman berikutnya}} \\
\endfoot

\hline
\endlastfoot

% ====== DATA TABEL DIMULAI DI SINI ======
NF.1 & Kinerja sistem (\textit{performance}) &
Sistem harus merespons halaman utama, presensi, dan \textit{feedback} dalam waktu $\leq$ 3 detik pada penggunaan normal, dan mampu menangani minimal 50 pengguna aktif bersamaan. \\
\hline
NF.2 & Keamanan dan privasi data &
Sistem menyediakan autentikasi per pengguna, pembatasan akses berdasarkan peran (\textit{role-based access}), serta komunikasi berbasis protokol aman (misalnya HTTPS) saat di-\textit{deploy}. Data siswa yang ditampilkan pada laporan harus dianonimkan bila bersifat agregat. \\
\hline
NF.3 & Kemudahan penggunaan (\textit{usability}) &
Antarmuka sistem harus sederhana dan konsisten, dapat diakses melalui peramban ponsel siswa tanpa instalasi aplikasi tambahan, serta menggunakan bahasa yang mudah dipahami oleh pengguna nonteknis. \\
\hline
NF.4 & Keandalan sistem (\textit{reliability}) &
Sistem harus memiliki \textit{availability} minimal 95\% dalam sebulan (tidak termasuk \textit{scheduled maintenance}) untuk memastikan akses yang konsisten. \\
\hline
NF.5 & Pemeliharaan dan pengembangan lanjut (\textit{maintainability}) &
Struktur kode harus modular sehingga fitur baru (misalnya kategori keluhan tambahan atau penyesuaian \textit{rating}) dapat dikembangkan tanpa perubahan besar pada sistem inti. \\
\hline
NF.6 & Kualitas fitur AI &
Analisis sentimen harus mencapai tingkat akurasi wajar (sekitar 75--80\%) pada \textit{dataset} uji internal, dan proses analisis per \textit{feedback} harus diselesaikan dalam waktu $\leq$ 2 detik pada lingkungan pengujian tugas akhir. \\
\hline
% ====== AKHIR DATA TABEL ======

\end{longtable}