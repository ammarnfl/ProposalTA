
\begin{longtable}{| p{0.08\textwidth} | p{0.26\textwidth} | p{0.58\textwidth} |}
\caption{\textit{Use case} sistem MBG \textit{Monitoring}}
\label{tab:bab4-usecase-sistem} \\
\hline
Kode & \textit{Use case} & Deskripsi \\
\hline
\endfirsthead

\multicolumn{3}{l}{\textit{Tabel \ref{tab:bab4-usecase-sistem} (lanjutan)}}\\
\hline
Kode & \textit{Use case} & Deskripsi \\
\hline
\endhead

\hline
\multicolumn{3}{r}{\textit{Bersambung ke halaman berikutnya}} \\
\endfoot

\hline
\endlastfoot

UC.1 & Login/Logout &
Semua aktor dapat melakukan \textit{login}/\textit{logout} ke dalam sistem, memverifikasi kredensial, dan memperoleh akses sesuai peran. \\
\hline
UC.2 & Mengelola menu harian &
Operator dapur menginput dan memperbarui menu harian beserta informasi porsi makanan yang akan disajikan kepada siswa. \\
\hline
UC.3 & Mencatat dan memperbarui distribusi makanan &
Operator dapur mencatat jadwal dan status distribusi (jumlah porsi keluar, porsi tersisa, dan status distribusi) yang digunakan sebagai sumber data untuk \textit{dashboard} distribusi. \\
\hline
UC.4 & Melihat informasi menu dan jadwal distribusi &
Siswa dan guru melihat menu makanan serta jadwal distribusi pada hari ini dan hari tertentu sehingga memperoleh informasi program MBG secara transparan. \\
\hline
UC.5 & Mengisi presensi konsumsi &
Siswa mencatat secara mandiri makanan yang dikonsumsi pada hari tersebut. \\
\hline
UC.6 & Memverifikasi dan melihat statistik presensi &
Guru memverifikasi atau mengoreksi presensi konsumsi bila diperlukan, serta melihat statistik presensi per kelas atau per hari untuk memantau partisipasi siswa. \\
\hline
UC.7 & Memberikan \textit{feedback} dan \textit{rating} makanan &
Siswa mengisi formulir \textit{feedback} teks dan memberikan \textit{rating} numerik terhadap makanan yang diterima (skala 1--5). \\
\hline
UC.8 & Melihat riwayat \textit{feedback} dan tindak lanjut &
Siswa melihat daftar \textit{feedback} dan \textit{rating} yang pernah dikirim beserta status tindak lanjutnya (misalnya diterima, dalam peninjauan, atau sudah ditindaklanjuti). \\
\hline
UC.9 & Mengirim observasi atau keluhan guru &
Guru menyampaikan observasi tambahan atau keluhan terkait pelaksanaan program, misalnya keterlambatan distribusi atau kualitas makanan. \\
\hline
UC.10 & Mengelola \textit{feedback} dan keluhan &
Operator dapur meninjau \textit{feedback} siswa dan keluhan guru, melakukan pengelompokan, menandai status penanganan, serta mencatat langkah perbaikan yang diambil. \\
\hline
UC.11 & Melihat \textit{dashboard} distribusi &
Operator dapur dan guru melihat ringkasan visual terkait distribusi makanan hari ini, seperti jumlah porsi yang sudah dibagikan, titik hambatan, dan status distribusi per kelas atau per sesi. \\
\hline
UC.12 & Melihat \textit{dashboard} sentimen dan \textit{rating} &
Operator dapur dan guru mengakses \textit{dashboard} yang menampilkan tren \textit{rating}, ringkasan sentimen, serta kategori keluhan utama dari waktu ke waktu sebagai dasar evaluasi kualitas program. \\
\hline
UC.13 & Menghasilkan laporan evaluasi berkala &
Sistem menyusun dan menyediakan laporan berkala (misalnya mingguan atau bulanan) terkait partisipasi konsumsi, \textit{rating} makanan, dan pola keluhan, yang dapat diunduh oleh operator dapur atau guru pembina. \\
\hline
\end{longtable}
